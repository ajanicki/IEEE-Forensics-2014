
Spoofing attacks are generally assumed to be performed at the microphone level.
In the case of voice conversion and speech synthesis, the practical scenario would then involve the recording of suitable data for the training or adaptation of conversion or synthesis systems, the fabrication of a spoofed utterance and then its presentation to the microphone of the ASV system.
None of the past work (including the authors') follows this process, preferring instead to emulate spoofing attacks by intervening after the microphone, immediately prior to feature extraction.
This approach is entirely justified in the case of linear transducers and channels, and mainly telephony scenarios in which spoofing can in any case be applied while bypassing the microphone~\cite{handbookChapter}.

%In keeping with the goals of this work, namely to compare the threat of replay with that of voice conversion and speech synthesis with strictly controlled protocols, all spoofing attacks are emulated in similar fashion as with the past work.
Voice conversion spoofing attacks are emulated with the approach described in Section~\ref{ssec:vconv}. 
A worst-case scenario is considered; conversion is performed with full prior knowledge of the ASV system, i.e.\ voice conversion is performed with exactly the same front-end processing as that used for ASV.
$H_y(f)$ and $H_x(f)$ use 19 LPCC and LPC coefficients respectively, with the former being obtained according to Eq.~\ref{eq:EMit} using 1024 Gaussian components.
Voice conversion is applied to the original impostor utterances which are converted towards the genuine speaker for any given trial.

Speech synthesis attacks are emulated according to the approach described in Section~\ref{ssec:spsyn} using an open source voice cloning toolkit\footnote{http://homepages.inf.ed.ac.uk/jyamagis/software/page37/page37.html} which is based on HTS.  It uses  a default configuration and default, standard speaker-independent models provided with the toolkit.  They are trained using American-English speech data from the Effective Multilingual Interaction in Mobile Environments (EMIME) corpus~\cite{Wester2010}.   
Adaptation data for each target speaker comprises three utterances (with transcriptions).  
For any given trial, speech synthesis spoofing attacks are generated using arbitrary text, thereby producing a spoofed utterance of duration close to that of the average test utterance.
