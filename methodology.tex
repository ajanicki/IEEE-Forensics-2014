
%{\bfseries to do later...}

%Due to a lack of real replay recordings 
%(e.g., similar to the MOBIO corpus collected in real environment~\cite{Khoury2013a})
This work aims to compare in a meaningful manner the threat of replay, speech synthesis and voice conversion spoofing attacks.  The goal places significant constraints on the choice of dataset, limiting the scope to those for which speech synthesis and voice conversion assessments have already been reported.  Large, standard datasets are also preferred.  

The work reported here was performed on a subset of the public corpora released in the context of the speaker recognition evaluation (SRE) campaigns administered by the National Institute for Standards and Technology (NIST)~\cite{ref to an evaluation plan for e.g. '08 data or just the web address to the NIST SRE evaluations}.  While the use of standard databases enables the comparison of results to others' work, it also necessitates artificial replay emulation.  While not ideal, this approach is preferred since it enables the comparison of results for replay, speech synthesis and voice conversion using otherwise identical protocols (and same source data).  Even if the approach is artificial, it is stressed that the work uses impulse responses calculated using real playback hardware and real acoustic environments.

%This work will also verify the effectiveness of replay detection, using two previously described replay countermeasures: 

%Case for emulated attacks - there is no alternative.  Similar to what has been done for all other work.

The following describes the ASV systems used in this study, the datasets, protocols and metrics, the approach to replay emulation and countermeasure implementations.