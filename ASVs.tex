
Since previous work in ASV anti-spoofing has shown different vulnerabilities for different ASV systems, this work is similarly performed with a range of representative technologies, from the standard to the state-of-the-art.
They include: 
(i) a standard Gaussian mixture model with universal background model (GMM-UBM) system;
(ii) a GMM supervector linear kernel (GSL) system;
(iii) a GSL system with nuissance atribute projection (NAP)~\cite{Campbell2006};
(iv) a GSL system with factor analysis (FA)~\cite{Fauve2007};
(v) a GMM-UBM system with factor analysis;
(vi) an iVector system~\cite{Dehak2011} (referred to from here on as IV), and 
(vii) an iVector system  with probabilistic linear discriminant analysis (PLDA)~\cite{Li2012} and length normalisation~\cite{Garcia2011} (referred to from here on as IV-PLDA). 
All seven ASV systems are tested with and without score normalisation.  
Symmetric normalisation (S-norm)~\cite{Kenny2010} is applied to IV and IV-PLDA systems while test normalisation (T-norm)~\cite{Auckenthaler2000} is used for the others. 

All ASV systems are implemented with the same LIA-SpkDet toolkit~\cite{Bonastre2008} and the ALIZE library~\cite{Bonastre2004} and stem from the work in~\cite{Fauve2007}.
They use a common UBM with 1024 Gaussian components and a common feature parametrisation: linear frequency cepstral coefficients (LFCCs), their deltas and delta energy. 
A speech activity detector is also common to each system.  
It fits a 3-component GMM to the log-energy distribution and adjusts the speech/non-speech threshold according to the GMM parameters~\cite{Bimbot2004}.
This approach has been used successfully in many independent studies, e.g.~\cite{magrin2001,fauve2008}. 
