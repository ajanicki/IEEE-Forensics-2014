Spoofing refers to the presentation of a falsified or manipulated sample 
to the sensor of a biometric system in order to provoke a high score and 
thus illegitimate acceptance.
In recent years, the automatic speaker verification (ASV) community has 
started to investigate spoofing and countermeasures 
actively~\cite{interspeechSpecialSession, handbookChapter}.  A growing 
body of independent work has now demonstrated the vulnerability of ASV 
systems to spoofing through 
impersonation~\cite{Blomberg2004,Farrus2008}, voice 
conversion~\cite{Perrot2005, Bonastre2007}, speech 
synthesis~\cite{Masuko1999, Leon2010} and attacks with non-speech, 
artificial tone-like signals~\cite{Alegre2012b}.

Common to the bulk of previous work is the consideration of attacks 
which require either specific skills, e.g.~impersonation, or high-level 
technology, e.g.~speech synthesis and voice conversion. With the 
noteworthy exceptions of~\cite{Lindberg1999,Villalba2010}, relatively 
little attention has been paid to low-effort spoofing attacks such as 
replay.  Replay attacks can be performed without any specific expertise 
nor any sophisticated equipment.  Since they are the most easily 
implemented, it is natural to assume that replay attacks will be the 
most commonly encountered in practice.  Nonetheless, the threat of 
replay attacks has neither been quantified using large, standard 
datasets nor compared to that of voice conversion or speech synthesis 
attacks.  This paper accordingly aims to re-assess ASV vulnerabilities 
to replay attacks using the same ASV systems and corpora used in 
previous assessments involving voice conversion and speech synthesis 
spoofing attacks. The effectiveness of reply countermeasures will be verified, too.  

The paper is organised as follows.  Section~2 describes previous research on reply attacks and the countermeasures against them, as well as an approach to 
simulate replay attacks in order that their effect can be compared to 
those of voice conversion and speech synthesis using the same corpora. Section~3 states the aim of the current work. A common experimental setup in which the vulnerabilities of seven different ASV systems is presented in Section~4. Results are presented in Section~5 and our conclusions and ideas for future work are presented in Section~6.
