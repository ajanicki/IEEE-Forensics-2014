
% COMMENT: you should refer to presentation attacks somewhere

Spoofing refers to the presentation of a falsified or manipulated sample 
to the sensor of a biometric system in order to provoke a high score and 
thus illegitimate acceptance.
In recent years, the automatic speaker verification (ASV) community has 
started to investigate spoofing and countermeasures 
actively~\cite{interspeechSpecialSession, handbookChapter, Wu2014a}. 
A growing 
body of independent work has now demonstrated the vulnerability of ASV 
systems to spoofing through 
replayed speech~\cite{Lindberg1999,Villalba2010},
impersonation~\cite{Blomberg2004,Farrus2008}, voice 
conversion~\cite{Perrot2005, Bonastre2007}, speech 
synthesis~\cite{Masuko1999, Leon2010} and attacks with non-speech, 
artificial, tone-like signals~\cite{Alegre2012b}.

Common to the bulk of previous work is the consideration of attacks 
which require either specific skills, e.g.~impersonation, or high-level 
technology, e.g.~speech synthesis and voice conversion. 
Only replay attacks can be performed with ease, requiring no specialist 
knowledge or expertise.  Since they are the most easily 
implemented, it is natural to assume that replay attacks will be the 
most commonly encountered in practice.  Nonetheless, the threat of 
replay attacks has neither been quantified using large, standard 
datasets nor compared to that of other attacks which, until now, have 
received considerably more attention in the literature.
With replay attacks being considerably the easiest to implement
and with discreet, high quality audio equipment now available to the masses,
it is the hypothesis in this article that replay attacks merit
considerably greater attention.
This paper accordingly aims to re-assess ASV vulnerabilities 
to replay attacks using the same ASV systems and corpora used in 
previous assessments involving voice conversion and speech synthesis 
spoofing attacks.  In addition the paper investigates the effectiveness of 
new countermeasures which aim to distinguish between genuine and replayed speech.  

The paper is organised as follows.  Section~2 describes previous research 
which has investigated various threats against ASV systems, including replay attacks and the countermeasures against them, and states the aim of the current work. A common experimental setup in which the vulnerabilities of seven different ASV systems are investigated, as well as the examined countermeasures are presented in Section~3. Results are presented in Section~4 and our conclusions and ideas for future works are presented in Section~5.
