
This paper assesses the threat to automatic speaker verification of replay spoofing attacks.  The threat is shown to be of least similar significance to that of voice conversion and speech synthesis.  The latter have attracted by far the greatest attention in the literature to date.  This paper argues that replay spoofing deserves greater attention; in contrast to voice conversion and speech synthesis, they require no specialist expertise or equipment and are therefore the most likely attack from a practical perspective.  

While there is potential to detect replay spoofing attacks, the approaches assessed in this paper rely upon the presence of measurable channel effects.  Approaches to channel compensation commonly used in today's state-of-the-art speaker verification systems can reduce channel effects thereby increasing the difficulty in detecting replay attacks.  Finally, replay attacks recorded with high-quality sound systems will be almost impossible to detect.  Alternative countermeasures will then be needed.  Challenge-response mechanisms are a suitable candidate but lack scientific and objective validation.

Lastly, the practical use of these and any other replay countermeasure will be dependent on the application.  It is extremely difficult to assess replay spoofing, hence the rather bold assumptions in this work and the use of emulated attacks.  While this is also a weakness of much of the past work, it is particularly true in the case of replay.  Further work will be needed to assess replay, and spoofing in general, with an application-driven methodology.  In all of these, there will be potential for replay attacks.  Since any security system is only as strong as its weakest link, replay is a genuine threat.