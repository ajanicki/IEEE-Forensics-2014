
This paper assesses the threat to automatic speaker verification of replay spoofing attacks.  The threat is shown to be of at least similar significance to that of speech synthesis and voice conversion.  The latter have attracted by far the greatest attention in the literature to date whereas replay spoofing is under-researched.  This paper argues that replay spoofing merits far greater attention; in contrast to speech synthesis and voice conversion attacks, replay attacks require no specialist expertise nor sophisticated equipment and are therefore the most likely attack in practice.  

While there is potential to detect replay spoofing attacks, the approaches assessed in this paper rely upon the presence of measurable distortion caused by the acoustic environment.  Approaches to channel compensation, universally popular in today's state-of-the-art speaker verification systems, can reduce channel effects thereby increasing the difficulty in detecting replay attacks.  In addition, replay attacks recorded with high-quality sound systems may be considerably more difficult to detect.  Alternative countermeasures will then be needed.  Challenge-response mechanisms are a suitable candidate but lack scientific and objective validation.

Lastly, the practical use of these and any other replay countermeasure are fundamentally dependent on the application.  It is extremely difficult to assess replay spoofing, hence some rather bold assumptions in this work and the use of replay emulation.  
%While similar assumptions and emulation methodologies typify much of the past work, it is particularly the case with all work related to replay.  
Further work will be needed to assess replay, and spoofing in general, with an application-driven methodology.  Whatever the methodology, however, since any security system is only as strong as its weakest link, countermeasures to thwart replay spoofing merit far greater attention than they have in the past.