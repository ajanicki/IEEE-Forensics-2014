
Despite the lack of attention to replay attacks in the literature and 
contrary to our hypothesis, results show that low-effort replay attacks pose a 
significant risk, surpassing that of comparatively high-effort attacks 
such as voice conversion and speech synthesis.  Worthy of note is the 
performance of the state-of-the-art iVector-PLDA system which, despite 
showing the best baseline performance, is the most vulnerable to replay 
attacks, especially for FARs below 10\%.

Future work should thus pay greater attention to replay attacks and, in 
particular, suitable replay attack countermeasures.  The assumption that higher-effort attacks pose the greatest threat might be ill-founded. Given that the implementation of replay attacks demands neither specific expertise nor any sophisticated equipment, the risk to ASV is arguably greater than that of voice conversion and speech synthesis which currently receive the most attention in the literature. Future evaluation should not only consider the threat of any particular attack, but also the ease with which they can be performed. We suggest that a risk-based approach should be adopted.
